\documentclass{article}
\usepackage{graphicx}
\usepackage[position=top]{subfig}
\usepackage[left=1in,right=1in,top=1in,bottom=1in]{geometry}
% ^^ moves subfigure titles to the top
\usepackage{pxfonts}


\def\rot#1{\raisebox{-1.35in}{\rotatebox{90}{#1}}}
% ^^ moves vertical titles down to the middle of the subfigures

\def\roto#1{\raisebox{-1.17in}{\rotatebox{90}{#1}}}



\begin{document}

\title{Supporting Materials for \textit{Feature-based and location-based volitional covert attention affect memory at different timescales}}

\author{Kirsten Ziman$^1$,
Madeline R. Lee$^1$,
Alejandro R. Martinez$^1$,\\
Ethan D. Adner$^2$,
and
Jeremy R. Manning\textsuperscript{$1, \dagger$}\\[0.1in]$^1$Department of Psychological and Brain Sciences and\\
$^2$Thayer School of Engineering,\\
Dartmouth College, Hanover, NH 03755\\
\textsuperscript{$\dagger$}Address correspondence to jeremy.r.manning@dartmouth.edu}

%\begin{titlepage}
  \maketitle
%\end{titlepage}

  \renewcommand{\thefigure}{S\arabic{figure}}
  
%\section*{Supplement}

%Here, we separately replicate the figures in our manuscript for the initial participant cohort and the replication cohort of each experiment.

% FIGURE - violin
\begin{figure}
\captionsetup[subfigure]{labelformat=empty}
\roto{\rlap{\textbf{Sustained}}}
 \subfloat[(a) Initial cohort]{
	\begin{minipage}[c][1\width]{
	   0.4\textwidth}
	   \centering
	   \includegraphics[width=\textwidth]{supp_figures/Maddy_sustain1violin.pdf}
	\end{minipage}}
 \hfill
  \subfloat[(b) Replication cohort]{
	\begin{minipage}[c][1\width]{
	   0.4\textwidth}
	   \centering
	   \includegraphics[width=\textwidth]{supp_figures/Maddy_sustain2violin.pdf}
	\end{minipage}}
 \hfill
  \subfloat{
	\begin{minipage}[c][1\width]{
	   0.1\textwidth}
	   \centering
	   \includegraphics[width=\textwidth]{supp_figures/violin_key.pdf}
	\end{minipage}}


% \hfill
  %\subfloat[Combined data]{
	%\begin{minipage}[c][1\width]{
	   %0.5\textwidth}
	   %\centering
	   %\includegraphics[width=\textwidth]{example-image-a.pdf}
	%\end{minipage}}



\roto{\rlap{\textbf{Variable}}}
  \subfloat[(c) Initial cohort]{
	\begin{minipage}[c][1\width]{
	   0.4\textwidth}
	   \centering
	   \includegraphics[width=\textwidth]{supp_figures/Maddy_variabl1violin.pdf}
	\end{minipage}}
 \hfill
  \subfloat[(d) Replication cohort]{
	\begin{minipage}[c][1\width]{
	   0.4\textwidth}
	   \centering
	   \includegraphics[width=\textwidth]{supp_figures/Maddy_variabl2violin.pdf}
	\end{minipage}}
 \hfill
  \subfloat{
	\begin{minipage}[c][1\width]{
	   0.1\textwidth}
	   \centering
	   \includegraphics[width=\textwidth]{supp_figures/violin_key.pdf}
	\end{minipage}}

\caption{\textbf{Familiarity ratings for attended, unattended, and
    novel stimuli.}  This figure follows the same formatting as
  Figure~2 in the main text, but the panels in this figure break down
  those results by participant cohort.  Each split violin plot displays the distribution of within-participant average familiarity ratings given to faces (left, darker colors) and scenes (right, lighter colors) during the memory phases of each experiment.  As shown in the legend (bottom), the colors indicate whether each image had been viewed during the presentation phase (\texttt{old}) or not; whether the given images matched the \texttt{attended category}; and/or whether the given images matched the \texttt{attended location}.  The colored lines above each set of violin plots denote statistical differences  (\textbf{pos}itive or \textbf{neg}ative differences in mean, collapsing over image category, assessed via two-tailed $t$-tests) between the distributions centered on the endpoints of each line.  The line thicknesses denote $p$-values as indicated in the legend.  Asterisks denote differences between the face versus scene distributions (assessed via two-tailed $t$-tests).}

\end{figure}





% FIGURE - gaze


\begin{figure}[ht]
\captionsetup[subfigure]{labelformat=empty}
\rot{\rlap{\textbf{Sustained}}}
  \subfloat[(a) Initial cohort]{
	\begin{minipage}[c][1\width]{
	   0.35\textwidth}
	   \centering
	   \includegraphics[width=\textwidth]{supp_figures/sustain_gaze_1.pdf}
	\end{minipage}}
 \hfill
  \subfloat[(b) Replication cohort]{
	\begin{minipage}[c][1\width]{
	   0.35\textwidth}
	   \centering
	   \includegraphics[width=\textwidth]{supp_figures/sustain_gaze_2.pdf}
	\end{minipage}}
 \hfill
  \subfloat{
	\begin{minipage}[c][1\width]{
	   0.2\textwidth}
	   \centering
	   \includegraphics[width=\textwidth]{supp_figures/gaze_legend.pdf}
	\end{minipage}}

\raisebox{.05in}{\rot{\rlap{\textbf{Variable}}}}
  \subfloat[(c) Initial cohort]{
	\begin{minipage}[c][1\width]{
	   0.35\textwidth}
	   \centering
	   \includegraphics[width=\textwidth]{supp_figures/variabl_gaze_1.pdf}
	\end{minipage}}
 \hfill
  \subfloat[(d) Replication cohort]{
	\begin{minipage}[c][1\width]{
	   0.35\textwidth}
	   \centering
	   \includegraphics[width=\textwidth]{supp_figures/variabl_gaze_2.pdf}
	\end{minipage}}
 \hfill
  \subfloat{
	\begin{minipage}[c][1\width]{
	   0.2\textwidth}
	   \centering
	   \includegraphics[width=\textwidth]{supp_figures/gaze_legend.pdf}
	\end{minipage}}

  \caption{\textbf{Horizontal coordinates of visual fixations.}  This figure follows the same formatting as
  Figure~3 in the main text, but the panels in this figure break down
  those results by participant cohort.  Each violin plot displays a distribution of participant-wise average horizontal gaze positions relative to a presented image.  The coordinates have been normalized such that a value of 1.0 denotes the furthest onscreen coordinate from the central fixation point \textit{towards} the direction of the given image, and -1.0 denotes the furthest onscreen coordinate from the central fixation point \textit{away from} the direction of the given image.  The gray bars in each panel mark the boundaries of the presented images.  The violin plots are broken down by participants' familiarity ratings during the memory phases of each experiment, and by each image's relation to the attention cue while that image appeared onscreen. }\end{figure}





% FIGURE - cued_window


\begin{figure}[ht]
\captionsetup[subfigure]{labelformat=empty}
\rot{\rlap{\textbf{Sustained}}}
  \subfloat[(a) Initial cohort]{
	\begin{minipage}[c][1\width]{
	   0.35\textwidth}
	   \centering
	   \includegraphics[width=\textwidth]{supp_figures/sust1_cued_window.pdf}
	\end{minipage}}
 \hfill
  \subfloat[(b) Replication cohort]{
	\begin{minipage}[c][1\width]{
	   0.35\textwidth}
	   \centering
	   \includegraphics[width=\textwidth]{supp_figures/sust2_cued_window.pdf}
	\end{minipage}}
 \hfill
  \subfloat{
	\begin{minipage}[c][1\width]{
	   0.2\textwidth}
	   \centering
	   \includegraphics[width=\textwidth]{supp_figures/red_key.pdf}
	\end{minipage}}



\raisebox{.05in}{\rot{\rlap{\textbf{Variable}}}}
  \subfloat[(c) Initial cohort]{
	\begin{minipage}[c][1\width]{
	   0.35\textwidth}
	   \centering
	   \includegraphics[width=\textwidth]{supp_figures/var1_cued_window.pdf}
	\end{minipage}}
 \hfill
  \subfloat[(d) Replication cohort]{
	\begin{minipage}[c][1\width]{
	   0.35\textwidth}
	   \centering
	   \includegraphics[width=\textwidth]{supp_figures/var2_cued_window.pdf}
	\end{minipage}}
 \hfill
  \subfloat{
	\begin{minipage}[c][1\width]{
	   0.2\textwidth}
	   \centering
	   \includegraphics[width=\textwidth]{supp_figures/red_key.pdf}
	\end{minipage}}

  \caption{\textbf{Familiarity ratings over time for images that matched cued image category.}  This figure follows the same formatting as
  Figure~4 (top panels) in the main text, but the panels in this figure break down
  those results by participant cohort.  Each curve reflects the average familiarity ratings for attended, unattended, and novel images (denoted in the legends on the right) within a succession of overlapping 20-image sliding windows.  Error ribbons denote 95\% confidence intervals, computed across participants. The paired horizontal lines at the bottom of each panel denote timepoints when the given pair of curves was statistically distinguishable (i.e., the topmost line color was statistically greater than the bottommost line color at $\alpha = 0.05$, via a paired two-tailed $t$-test.)}
  \label{fig:familiarity_timecourse_category3}
\end{figure}







% FIGURE - uncued_window


\begin{figure}[ht]
\captionsetup[subfigure]{labelformat=empty}
\rot{\rlap{\textbf{Sustained}}}
  \subfloat[(a) Initial cohort]{
	\begin{minipage}[c][1\width]{
	   0.35\textwidth}
	   \centering
	   \includegraphics[width=\textwidth]{supp_figures/sust1_uncued_window.pdf}
	\end{minipage}}
 \hfill
  \subfloat[(b) Replication cohort]{
	\begin{minipage}[c][1\width]{
	   0.35\textwidth}
	   \centering
	   \includegraphics[width=\textwidth]{supp_figures/sust2_uncued_window.pdf}
	\end{minipage}}
 \hfill
  \subfloat{
	\begin{minipage}[c][1\width]{
	   0.2\textwidth}
	   \centering
	   \includegraphics[width=\textwidth]{supp_figures/blue_key.pdf}
	\end{minipage}}

\raisebox{.05in}{\rot{\rlap{\textbf{Variable}}}}
  \subfloat[(c) Initial cohort]{
	\begin{minipage}[c][1\width]{
	   0.35\textwidth}
	   \centering
	   \includegraphics[width=\textwidth]{supp_figures/var1_uncued_window.pdf}
	\end{minipage}}
 \hfill
  \subfloat[(d) Replication cohort]{
	\begin{minipage}[c][1\width]{
	   0.35\textwidth}
	   \centering
	   \includegraphics[width=\textwidth]{supp_figures/var2_uncued_window.pdf}
	\end{minipage}}
 \hfill
  \subfloat{
	\begin{minipage}[c][1\width]{
	   0.2\textwidth}
	   \centering
	   \includegraphics[width=\textwidth]{supp_figures/blue_key.pdf}
	\end{minipage}}

  \caption{\textbf{Familiarity ratings over time for images that matched uncued image category.}  This figure follows the same formatting as
  Figure~4 (bottom panels) in the main text, but the panels in this figure break down
  those results by participant cohort.  Each curve reflects the average familiarity ratings for attended, unattended, and novel images (denoted in the legends on the right) within a succession of overlapping 20-image sliding windows.  Error ribbons denote 95\% confidence intervals, computed across participants. The paired horizontal lines at the bottom of each panel denote timepoints when the given pair of curves was statistically distinguishable (i.e., the topmost line color was statistically greater than the bottommost line color at $\alpha = 0.05$, via a paired two-tailed $t$-test.)}
  \label{fig:familiarity_timecourse_category4}
\end{figure}




% FIGURE - cued_loc


\begin{figure}[ht]
\captionsetup[subfigure]{labelformat=empty}
\rot{\rlap{\textbf{Sustained}}}
  \subfloat[(a) Initial cohort]{
	\begin{minipage}[c][1\width]{
	   0.35\textwidth}
	   \centering
	   \includegraphics[width=\textwidth]{supp_figures/sust1_cued_loc.pdf}
	\end{minipage}}
 \hfill
  \subfloat[(b) Replication cohort]{
	\begin{minipage}[c][1\width]{
	   0.35\textwidth}
	   \centering
	   \includegraphics[width=\textwidth]{supp_figures/sust2_cued_loc.pdf}
	\end{minipage}}
 \hfill
  \subfloat{
	\begin{minipage}[c][1\width]{
	   0.2\textwidth}
	   \centering
	   \includegraphics[width=\textwidth]{supp_figures/red_blue_key.pdf}
	\end{minipage}}


\raisebox{.05in}{\rot{\rlap{\textbf{Variable}}}}
  \subfloat[(c) Initial cohort]{
	\begin{minipage}[c][1\width]{
	   0.35\textwidth}
	   \centering
	   \includegraphics[width=\textwidth]{supp_figures/var1_cued_loc.pdf}
	\end{minipage}}
 \hfill
  \subfloat[(d) Replication cohort]{
	\begin{minipage}[c][1\width]{
	   0.35\textwidth}
	   \centering
	   \includegraphics[width=\textwidth]{supp_figures/var2_cued_loc.pdf}
	\end{minipage}}
 \hfill
  \subfloat{
	\begin{minipage}[c][1\width]{
	   0.2\textwidth}
	   \centering
	   \includegraphics[width=\textwidth]{supp_figures/red_blue_key.pdf}
	\end{minipage}}

\caption{\textbf{Familiarity ratings over time for images that matched the cued image location.}  This figure follows the same formatting as
  Figure~5 (top panels) in the main text, but the panels in this figure break down
  those results by participant cohort.}
\label{fig:familiarity_timecourse_location5}
\end{figure}







% FIGURE - uncued_loc


\begin{figure}[ht]
\captionsetup[subfigure]{labelformat=empty}
\rot{\rlap{\textbf{Sustained}}}
  \subfloat[(a) Initial cohort]{
	\begin{minipage}[c][1\width]{
	   0.35\textwidth}
	   \centering
	   \includegraphics[width=\textwidth]{supp_figures/sust1_uncued_loc.pdf}
	\end{minipage}}
 \hfill
  \subfloat[(b) Replication cohort]{
	\begin{minipage}[c][1\width]{
	   0.35\textwidth}
	   \centering
	   \includegraphics[width=\textwidth]{supp_figures/sust2_uncued_loc.pdf}
	\end{minipage}}
 \hfill
  \subfloat{
	\begin{minipage}[c][1\width]{
	   0.2\textwidth}
	   \centering
	   \includegraphics[width=\textwidth]{supp_figures/orange_blue_key.pdf}
	\end{minipage}}



\raisebox{.05in}{\rot{\rlap{\textbf{Variable}}}}
  \subfloat[(c) Initial cohort]{
	\begin{minipage}[c][1\width]{
	   0.35\textwidth}
	   \centering
	   \includegraphics[width=\textwidth]{supp_figures/var1_uncued_loc.pdf}
	\end{minipage}}
 \hfill
  \subfloat[(d) Replication cohort]{
	\begin{minipage}[c][1\width]{
	   0.35\textwidth}
	   \centering
	   \includegraphics[width=\textwidth]{supp_figures/var2_uncued_loc.pdf}
	\end{minipage}}
 \hfill
  \subfloat{
	\begin{minipage}[c][1\width]{
	   0.2\textwidth}
	   \centering
	   \includegraphics[width=\textwidth]{supp_figures/orange_blue_key.pdf}
	\end{minipage}}

\caption{\textbf{Familiarity ratings over time for images that matched uncued image location.}  This figure follows the same formatting as
  Figure~5 (bottom panels) in the main text, but the panels in this figure break down
  those results by participant cohort.}
\label{fig:familiarity_timecourse_location6}
\end{figure}



% FIGURE - novel window


\begin{figure}[ht]
\captionsetup[subfigure]{labelformat=empty}
\rot{\rlap{\textbf{Sustained}}}
  \subfloat[(a) Initial cohort]{
	\begin{minipage}[c][1\width]{
	   0.35\textwidth}
	   \centering
	   \includegraphics[width=\textwidth]{supp_figures/sust1_novel_window.pdf}
	\end{minipage}}
 \hfill
  \subfloat[(b) Replication cohort]{
	\begin{minipage}[c][1\width]{
	   0.35\textwidth}
	   \centering
	   \includegraphics[width=\textwidth]{supp_figures/sust2_novel_window.pdf}
	\end{minipage}}
 \hfill
  \subfloat{
	\begin{minipage}[c][1\width]{
	   0.2\textwidth}
	   \centering
	   \includegraphics[width=\textwidth]{supp_figures/novel_key.pdf}
	\end{minipage}}

\raisebox{.05in}{\rot{\rlap{\textbf{Variable}}}}
  \subfloat[(c) Initial cohort]{
	\begin{minipage}[c][1\width]{
	   0.35\textwidth}
	   \centering
	   \includegraphics[width=\textwidth]{supp_figures/var1_novel_window.pdf}
	\end{minipage}}
 \hfill
  \subfloat[(d) Replication cohort]{
	\begin{minipage}[c][1\width]{
	   0.35\textwidth}
	   \centering
	   \includegraphics[width=\textwidth]{supp_figures/var2_novel_window.pdf}
	\end{minipage}}
 \hfill
  \subfloat{
	\begin{minipage}[c][1\width]{
	   0.2\textwidth}
	   \centering
	   \includegraphics[width=\textwidth]{supp_figures/novel_key.pdf}
	\end{minipage}}

   \caption{\textbf{Familiarity ratings over time for attended-category and unattended-category novel images.}  This figure follows the same formatting as
  Figure~6 (top panels) in the main text, but the panels in this figure break down
  those results by participant cohort.}\end{figure}



% FIGURE - diff


\begin{figure}[ht]
\raisebox{-.4in}{\rot{\rlap{\textbf{Sustained}}}}
  \subfloat[Initial cohort]{
	\begin{minipage}[c][1\width]{
	   0.5\textwidth}
	   \centering
	   \includegraphics[width=\textwidth]{supp_figures/sust1_diff.pdf}
	\end{minipage}}
 \hfill
  \subfloat[Replication cohort]{
	\begin{minipage}[c][1\width]{
	   0.5\textwidth}
	   \centering
	   \includegraphics[width=\textwidth]{supp_figures/sust2_diff.pdf}
	\end{minipage}}


\raisebox{-.35in}{\rot{\rlap{\textbf{Variable}}}}
  \subfloat[Initial cohort]{
	\begin{minipage}[c][1\width]{
	   0.5\textwidth}
	   \centering
	   \includegraphics[width=\textwidth]{supp_figures/var1_diff.pdf}
	\end{minipage}}
 \hfill
  \subfloat[Replication cohort]{
	\begin{minipage}[c][1\width]{
	   0.5\textwidth}
	   \centering
	   \includegraphics[width=\textwidth]{supp_figures/var2_diff.pdf}
	\end{minipage}}

 \caption{\textbf{Familiarity ratings over time for attended-category and unattended-category novel images.}  This figure follows the same formatting as
  Figure~6 (bottom panels) in the main text, but the panels in this figure break down
  those results by participant cohort.}
\end{figure}












\end{document}
